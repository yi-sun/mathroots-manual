\documentclass[10pt]{amsart}
\usepackage[margin=1in]{geometry}
\usepackage{amssymb, amsmath, url}

\begin{document}
\title{MathROOTS Academic Manual: v1}
\author{Yi Sun}
\email{yi.sun@post.harvard.edu}
\date{\today}
\begin{abstract}
This is a guide to running the academic side of MathROOTS.  v1 July 2016.
\end{abstract}
\maketitle

\section{Before MathROOTS}

\subsection{Hire the academic team (February)}

In 2015--2016, the academic team was composed of an Academic Coordinator and Head Mentor who gave lectures and 5 Academic Mentors for 20 students.  You should discuss possible changes depending on the budget and target size of the program with Pavel and Slava.  In the past, 1-2 of the academic mentors also served as Residential Counselors and lived in the dorm with the students, which was extremely helpful for getting student feedback.  Depending on the hiring constraints for the year, you should email a short blurb to (1) mailing lists (MIT undergraduate math majors and graduate students) and (2) individual students who you want to target.  It's best if at least a few of the Academic Mentors are students of color.  You should ask applicants to submit resumes, interview the applicants together with Tanya, and make offers to the staff by mid-March.

A sample blurb from 2015 is below.

\begin{center}
\begin{verbatim}
The MIT PRIMES program is seeking undergraduates or recent graduates
with strong backgrounds in mathematics to serve as Academic Mentors in
the new MathROOTS summer camp for gifted minority high school
students.   The camp will be directed by Dr. Slava Gerovitch and
advised by Prof. Pavel Etingof.  It will teach problem solving skills
and select teams of students to compete in the Ibero-American and
Pan-African Mathematics Olympiads this summer.

RESPONSIBILITIES: Mentors will run problem-solving sessions for small
groups of students, give feedback on student papers, and organize
mathematical activities.

DATES: June 14, 2015 to June 24, 2015.

COMPENSATION: Mentors will receive $3000 and the possibility of
on-campus housing for the duration of the program.

If you are interested in becoming a mentor for MathROOTS, contact
Tanya Khovanova (tanya@math.mit.edu) or Yi Sun (yisun@math.mit.edu) by
March 1.
\end{verbatim}
\end{center}

\subsection{Open and publicize the application (February)}

At the same time, you should work with the Program Director to open the application on the MathROOTS website.  The sooner this is done, the better, in order to allow more students to learn about the program and apply.  In 2015 and 2016, the deadline was April 1.  You should then ask MIT Admissions to help publicize the application; you can also help publicize it at venues such as AoPS (\url{www.aops.com}) or BEAM (\url{www.beammath.org}).  

\subsection{Select students (April)}

After the application is closed, you can select students to admit.  It is important to do this early, since students may have conflicting summer plans.  The process is as follows:

\begin{itemize}
\item Ask MIT Admissions to do a rough screen of applications by eligibility and basic criteria.  This will reduce the numbers from 200+ to 40.

\item Work with Tanya to select 20 admits and a waitlist of 5 from these 40 students.  The entire waitlist might be used in some years.

\item Give this list to the Program Director to (1) send admissions offers, (2) verify student information with schools, and (3) arrange logistics.  This process will often take 3-4 weeks.
\end{itemize}

\subsection{Invite guest lecturers (May)}

You should invite 4-5 guest lecturers to give evening lectures on different topics in math throughout the program.  In 2015 and 2016, these were chosen from faculty, postdocs, and graduate students in the Boston area.  The speakers for 2016 were: Professor Chelsea Walton (Temple), Professor Jelani Nelson (Harvard), Professor Oscar Fernandez (Wellesley), and Dr. Bobby Wilson (MIT).  I invited the speakers by email, and every speaker I invited accepted. Some of the guest lecturers should be people of color, as they are often strong role models for the students.  You should also prepare a small gift for the speaker; in 2016, we gave speakers glass Klein bottles from \url{www.kleinbottle.com}.

\subsection{Send out a diagnostic quiz (May)}

Once the list of students is finalized, you should email them welcoming them to the program and asking them to complete a diagnostic quiz, submitted by email.  This will give you a better sense of the students' strengths and weaknesses and provide some material for discussion at the beginning of the program.

\begin{center}
\begin{verbatim}
Hi all,

My name is Yi Sun, and I will be Academic Coordinator for MathROOTS 2016.  I am very 
excited to welcome all of you to MathROOTS!

Before you arrive, we'd like to get to know your mathematical interests and style a 
bit better.  To this end, we've prepared a quiz (attached) for you to work on over 
the next month.   This quiz is not to assess you but to help us better tailor the 
camp curriculum to your interests.  If you have any questions or would like to 
discuss the quiz, please don't hesitate to send me an email.

Please take Part 1 of the quiz under a 1 hour time limit, but feel free to take as 
much time as you'd like on Part 2.  Once you are done, scan your solutions and 
e-mail them  to yisun@math.mit.edu by June 7. (If you have difficulty accessing a 
scanner, feel free to type or take pictures of your solutions.)

We hope you enjoy the quiz, and we look forward to meeting you next month!

Best,
Yi
\end{verbatim}
\end{center}

\subsection{Plan the curriculum and schedule (May)}

The academic curriculum consists of lectures, problem sessions, intro/outro tests, proof-based tests, test reviews, guest lectures, and special sessions for proof writing and a team contest.  You should coordinate with Tanya to plan the topics and sequence of the lectures and problem sessions.  The topics and schedule from 2016 can be seen at \url{mathroots.mit.edu/schedule.html}.  Things to keep in mind are:
\begin{itemize}
\item Schedule a dedicated session for proof writing on one of the first few days of the program.

\item Increase the technical difficulty of the topics as the program progresses.

\item Make sure that induction is one of the first few lectures, as it is a necessary prerequisite for many other topics.

\item Schedule the team contest as the last event of the last day.

\item Schedule the intro test for the first day and the outro test for the last day.

\item Schedule at least one test-free day between each test, and try to schedule tests in the morning so that students are more fresh.

\item Allocate one dedicated review session for each test.

\item In 2016, students felt there was not enough problem session time.
\end{itemize}
At this time, you will also want to prepare handouts and problem sets for each lecture. For lectures, keep in mind that the level of the students will vary greatly; I think it is most beneficial to focus on the weaker students, while having some problems to challenge the stronger ones.  The problem sets will be used by Academic Mentors in the problem sessions after the lectures and should also contain problems of varying levels; 10 problems per lecture has been a reasonable number, possibly a bit high.  Finally, you will assign Academic Mentors to each session; they will run the problem sessions (3 per session), proctor the tests (1 per test), and run the test reviews (3 per review).  You can also ask the mentors if they would like to optionally give an evening lecture or take one of the lecture slots.  This gives the students more variety in their instruction.

\subsection{Prepare the team contest (May)}

You should ask one Academic Mentor to take charge of organizing the other mentors to prepare the team contest; a good starting point is the contest from the previous year.  This should be 20 questions, usually involving simple proofs.  This mentor should also take care of asking students to form 4 teams of 5 and distributing the team contest to students on the first or second day of the program.

\subsection{Prepare tests (May)}

You should prepare 5 tests.  The Intro and Outro tests are intended to assess the students' aggregate improvement over the course of the program, so they should be as similar as possible, allow for a reasonable range of scores, and correspond to an outside metric.  In 2016, we used the 2003 AMC12A and AMC12B, and I would recommend continuing to use the AMC12 exam from different years.  You can find the questions online at \url{www.aops.com}.

The three proof-based exams should increase in difficulty.  They are of differing lengths: Test 1 (2 hours, 4 problems), Test 2 (2 hours, 4 problems), Test 3 (4 hours, 5 problems).  Each test should contain at least 1 problem (preferably 2) which students can try examples or make partial progress on if they are stuck.  You should also check that no problem uses a key concept that the students have not yet been exposed to.  The tests from previous years are a good start.

\subsection{Set up logistics (May)}

You should create the following and share it with the academic team:
\begin{itemize}
\item A public website containing the schedule viewable by students and staff.

\item A Google group for academic staff.

\item A Dropbox folder containing all academic documents (lecture handouts, problem sets, tests, team contest, diagnostic test solutions).

\item A Google spreadsheet to aggregate student scores on tests.

\item A Slack group for all staff.
\end{itemize}

\subsection{Grade the diagnostic quiz (June)}

Students will email you their solutions to the diagnostic quiz, which you should store in Dropbox.  You should organize the Academic Mentors to grade the diagnostic quiz in preparation for a review session on one of the first few days of the program.  These can also be good examples for a proofwriting session.

\section{During MathROOTS}

\subsection{Daily schedule}

In 2016, the daily schedule was:
\begin{itemize}
\item AM1 (09:00 to 10:20): Lecture
\item AM2 (10:30 to 12:30): Problem Session (or test)
\item PM1 (14:30 to 15:50): Lecture
\item PM2 (16:00 to 18:00): Problem Session
\item PM3 (19:00 to 20:00): Guest Lecture (some days only)
\end{itemize}
Students felt that PM2 ended a bit late, so it may be worth shifting the whole schedule earlier by 30 minutes.

\subsection{Test grading}

After each test, you should assign each problem to an Academic Mentor to grade.  Each solution is assigned two grades, math and style.  Math is out of 7, corresponds to the olympiad scale, and assesses the correctness of the solution.  Style is out of 1, with 0.8 being average, and assesses the quality of the mathematical writing and presentation.  After grading is complete, you should assign an Academic Mentor to be in charge of aggregating and collating all student papers and returning collated tests to the students.   It is important to emphasize to students that they write each solution on a different sheet of paper.

\subsection{Team contest}

You should run team contest as a 4-team tournament as follows:
\begin{itemize}
\item Have 2 teams in each room, and two rounds.  The winners and losers of the first round will meet in the second round.  Round 1 will use Problems 1-10 and Round 2 will use Problems 11-20.
\item Have 3 judges in each room.
\item In each round, the teams will present in a snake fashion, i.e. 1, 2, 2, 1, 1, 2, 2, etc.  The first two problems from each team will be worth 4 points, the next two 5, and so on up to 8.
\item When it is their turn to present, the team will select a member to present who has not yet presented and choose a not-yet-presented problem to present a solution to.  They will have 4 minutes to present a complete solution, after which judges will deliberate and give a score.  At the judges' discretion, problems which are not solved may be returned to the pool for future presentation by the other team only.
\end{itemize}

\subsection{Guest lectures}

You should try to schedule guest lectures on alternate days at 7:00pm.  Before each guest lecture, send the lecturer a reminder email and meet them in the lecture room 5-10 minutes before to deal with any logistics.  You can also invite the guest lecturers to dinner with the students beforehand.

\subsection{Collect comments about students}

Starting in the second week of the program, you should organize the academic team to write observations about the students to be used in college recommendation letters after the end of the program.  These can be easily aggregated in a Google spreadsheet.

\section{After MathROOTS}

\subsection{Write recommendation letters}

You should organize the academic team to write college recommendation letters for the students.  Assign each team member to 3-4 student for whom to write a 1-2 paragraph recommendation aggregating all observations made.  Things to discuss include (1) performance in classes and problem sessions, (2) test performance, (3) interaction with students outside of class sessions, and (4) improvement over the course of the program.  You will then combine these paragraphs with a standard paragraph giving some background and context about the program and work with the Program Director to send them out on behalf of the students over the course of the next year.

\subsection{Update this guide}

Update this guide with improvements and suggestions for things to do differently next year!

\section{Ideas for improvement or experimentation}

\begin{itemize}
\item \textbf{(2016) Collect student feedback:}  It might be useful to give the students an online survey to collect feedback directly from them about the program.

\item \textbf{(2016) Raise more money:} I contacted some potential corporate sponsors in 2016 and was told to ask again in a future year for a smaller amount.

\item \textbf{(2016) Increase program size:} With more funding and more effort on student recruitment, it would be nice to increase the number of students.  This would give us the ability to divide into two classes of differing ability levels, which enable us to target lectures to the level of the students in a more fine-grained way.

\item \textbf{(2016) Create a lead mentor role:} It would be nice to put one Academic Mentor in charge of many of the logistical tasks listed here (running team contest, running test grading) to enable you to spend more time thinking about the curriculum.

\item \textbf{(2016) Communicate better about what was covered:}  It would be nice to have a common channel to record what was done in each lecture and problem session and any notes about how to adjust for later sessions.  We used email in 2016, but I think something like Slack might work better.
\end{itemize}

\end{document}
